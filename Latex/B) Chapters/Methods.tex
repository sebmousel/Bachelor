\chapter{Methods}

\section{Implementing Code}

As mentioned, all calculations are computed with the programming language Python within Virtual Studio Code.
To run the code, the libraries numpy, scipy.linalg and matplotlib are used.

First off, the partial transpose operation was defined with the help of the function np.transpose().
(insert pic of line)\\
Then the function is\_PPT() uses the \verb|partial_transpose()| and the scipy.linalg function eigenvalsh() to find the eigenvalues
that are then checked for non-positive values.
(insert function)\\
To compute the realignement criterion, we use the realign() function that returns the realigned matrix. SVD() returns the singular values of said matrix.
Finally, ralign\_log() calculates the $\log_2$ of the sum of the found singular values. This is then evaluated in realign\_crit().

generate\_bell\_states() gives us the Bell states a bipartite system of qutrits and rhob() defines the state $\rho_b$ used in this work.

Then lorentz\_boost\_k() and lam\_boost() respectively define the Lorentz boosts $L_k$ and $\Lambda(\textbf{e},\xi)$.\\
wigner() returns the Wigner rotation and D() the three dimensional, unitary, irreducible representation of the rotation group.??\\

The functions boosted\_state1() and boosted\_state2() both compute the state after the Lorentz boost. While boosted\_state1() only considers the state in the form
$\ket{\psi_1^{mom}} = \frac{1}{\sqrt{2}}(\ket{k_1,k_2} + \ket{k_2,k_1})$, boosted\_state() considers the generalized case $\ket{\psi_2^{mom}} = \cos{\theta}\ket{k_1,k_2} + \sin{theta}\ket{k_2,k_1}$

To find the values for the realignment criterion, we use realign\_val\_12\_21() with boosted\_state1() and realign\_val\_theta2() with boosted\_state2().

The function realign\_general() is used for the general state case p*sum1 + (1-p)*sum2.\cite{horodecki1998mixed}\cite{hiesmayr2024bipartite}

\lstinputlisting[language=Python]{../Code/Func_Col.py}

%talk about reproducing results
To reproduce the result from the paper, we simply use the function \verb|realign_val_21_12(e,xi,mom1,mom2)|.
This function takes the direction $e$, the rapitidity $xi$, and momenta $mom1$ and $mom2$. Then it calculates 1000
values of realignment in the interval 0 to $\frac{1}{3}$. It uses \verb|boosted_state1| defined by cite equation to boost
the state $\rho_b$, \verb|realign_log| to calculate the realignment (cite equation) and \verb|is_PPT| to check if the matrix 
satisfies the PPT criterion. This is then plotted using the following code: insert code

%expand upon these results
Now we consider the case where the probability to find the momenta of the 2 particles isn't equal. To do this, lets make the probability
dependant of $\theta$, so that finding the momenta in state $ket{k_1,k_2}$ has a probability of $\cos{\theta}$ and in state $\ket{k_2,k_1}$
a probability of $\sin{\theta}$. The whole state can be written as:

\begin{equation}
    \ket{\Psi_mom} = \cos{\theta}*\ket{k_1,k_2} + \sin{\theta}*\ket{k_2,k_1}
\end{equation}

\verb|boosted_state2| takes the additional argument \verb|theta|, in comparison to \verb|boosted_state1| and iterates through 1000 states from 
the x value 0 to $\frac{1}{3}$ which has the same procedure as \verb|boosted_state1| with the additional argument \verb|theta|. Then 
\verb|realign_val_theta2| calculates the realignment for each state. This is plotted using this code: insert code.

To explore the behaviour of bound entangled states, we compute the states boosted in different direction. The direction will be denoted in the form
\verb|[x,y,z]|. For example, [1,0,0] stands for a boost in x direction. This will be done for [1,0,0],[0,1,0],[1,1,1],[1,1,0],[1,0,1],[0,1,1]. 
Again, using \verb|realign_val_12_21| and by iterating through all direction, we find values for the realignment and check the PPT criterion. Then 
the values are plotted using the following code: insert code.

Lastly, when considering the general state, we generate the state $\rho_2$ that is defined as:

\begin{equation}
    \rho_2 = \ket{k_1,k_2}\bra{k_1,k_2} \otimes \ket{\Phi_{spin}}\bra{\Phi_{spin}}
\end{equation} with

\begin{equation}
    \ket{\Phi_{spin}} = \sum_{k,l=0}^{2} a_{k,l}\ket{\Omega_{k,l}}
\end{equation} and

\begin{equation}
    a_{k,l} =\begin{pmatrix}
            0 & 2/9 & 2/9 \\
            0 & 2/9 & 1/18 \\
            5/18 & 0 & 0 
            \end{pmatrix}
\end{equation}

Then \verb|realign_general| computes the respective boost and returns the values for the realignment and checks PPT criterion.
This is plotted using the following code: insert code