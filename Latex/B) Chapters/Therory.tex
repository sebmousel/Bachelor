\chapter{Theory}

\section{Quantum States}

Pure states:

\begin{equation}
    \rho = \ket{\psi}\bra{\psi}
\end{equation}

Mixed States:

\begin{equation}
    \rho = \sum_{i} p_i \ket{\psi_i}\bra{\psi_i}
\end{equation}

General States:

\begin{itemize}
    \item Tr($\rho$) <= 1
    \item $\rho$ >= 1
\end{itemize}

\section{Bipartite States}

Separability: $\rho_{AB} = \sum_{i} p_i \rho_A \otimes \rho_B$ and $\sum_{i} p_i = 1$

For 2$\otimes$2 and 2$\otimes$3, states can be described using the Peres-Horodecki Positive Partial Transpose criterion, short PPT.
In these lower dimensions, PPT is sufficient and necessary to characterize a quantum state.\\

For higher dimensions, PPT is no longer sufficient. There arises the existence of entangled states that are also positive semidefinite, which are called bound entangled states.
For these cases, additional criterions can be used to help characterize these quantum states. One of these criterion is the realignment criterion. First we define:

\begin{equation}
    \text{Realignment}(\rho_{AB}) = \log_2({\sum_{i}\sigma_i(\tilde{\rho}_{AB})})
\end{equation}

Here $\tilde{\rho}$ represents the realigned densitiy matrix $\rho_{AB}$ (cite the paper for realignment). $\sum_i\sigma_i(\tilde{\rho}_{AB})$ are the singular values of the realigned matrix.
It says that if the value of Realignment($\rho_{AB}$) > 0, then this state is entangled. However there exist entangled states that fulfill Realignment($\rho$) < 0.

Entanglement Witness

\section{Magic simplex}

\begin{equation}
    \mathcal{M}_3 \equiv \left\{\rho = 2 \sum_{k,l=0} c_{k,l} P_{k,l} \mid 2 \sum_{k,l=0} c_{k,l} = 1, \, c_{k,l} \geq 0\right\}
\end{equation}

$P_{k,l}$ = $\ket{\Omega_{k,l}}\bra{\Omega_{k,l}}$, $\ket{\Omega_{k,l}}$ being the bell states that can be generated from $\ket{\Omega_{0,0}} = \frac{1}{\sqrt{3}}(\ket{00} + \ket{11} + \ket{22})$ 
using the Weyl-operator. $\ket{\Omega_{k,l}} = W_{k,l} \otimes \mathbb{1}_3 \ket{\Omega_{0,0}}$. Here $W_{k,l} \equiv \sum_{j=0}^{2}\omega^{j\cdot k}\ket{j}\bra{j+l}$ and $\omega = e^{\frac{2\pi\imath}{3}}$.

\section{Lorentz Boost}

\begin{equation}
    \braket{k,\sigma}{k',\sigma'} = 2 k^0 \delta^3(\textbf{k} - \textbf{k'} \cdot \delta_{\sigma\sigma'})
\end{equation}

\begin{equation}
    L_k = \frac{1}{m}\begin{pmatrix}
        k^0 & \textbf{k}^T \\
        \textbf{k} & m \mathds{1}_3+\frac{\textbf{k}\otimes \textbf{k}^T}{m+k^0} 
        \end{pmatrix}
\end{equation}

\begin{equation}
    U(\Lambda)\ket{k,\sigma} = \mathcal{D}_{\lambda\sigma}(R(\Lambda,k))\ket{\Lambda k,\lambda}
\end{equation}

\begin{equation}
    \mathcal{D}(R) = V \vdot R \vdot V^\dagger, \quad V^\dagger\vdot V = \mathds{1}_3
\end{equation}

\begin{equation}
    V = \frac{1}{\sqrt{2}}\begin{pmatrix}
        -1 & i & 0 \\
        0 & 0 & \sqrt{2} \\
        1 & i & 0 
        \end{pmatrix}
\end{equation}

\begin{equation}
    \rho = \sum \rho_{\sigma \lambda, \sigma' \lambda'} \ket{k_m, k_n ; \sigma, \lambda} \bra{k_{m'}, k_{n'} ; \sigma', \lambda'}
\end{equation}

\begin{equation}
    \rho' = \left[ U(\Lambda(e, \xi)) \otimes U(\Lambda(e, \xi)) \right] \rho \left[ U(\Lambda(e, \xi)) \otimes U(\Lambda(e, \xi)) \right]^\dagger
\end{equation}

\begin{equation}
    \Lambda(e, \xi) =
    \begin{pmatrix}
        \cosh \xi & e^T \sinh \xi \\
        e \sinh \xi & \mathds{1}_3 + (\cosh \xi - 1) e \otimes e^T
    \end{pmatrix}
\end{equation}

\begin{equation}
    \rho = \ket{\psi^{mom}}\bra{\psi^{mom}} \otimes (\sum_{i} p_i\ket{\phi_i^{spin}}\bra{\phi_i^{spin}}) \equiv \ket{\psi^{mom}}\bra{\psi^{mom}} \otimes \rho_{spin}
\end{equation}

\begin{equation}
    \ket{\psi^{mom}} = \sum_{i,j=1}^{2}a_{ij}\ket{k_i,k_j}, \quad \sum_{i,j=1}^{2}|a_{ij}|^2 = 1.
\end{equation}

\begin{equation}
    \rho^\prime_{spin} = \text{Tr}^{mom}(\rho^\prime) = \sum_{i,j=1}^{2}|a_{ij}|[D^T(\Lambda,k_i) \otimes D^T(\Lambda,k_j)]\rho_{spin}[D^*(\Lambda,k_i) \otimes D^*(\Lambda,k_j)]
\end{equation}

Lorentz Boost
